% -*- ispell-dictionary: french; -*-
\documentclass[10pt]{beamer}
\usepackage{fontspec}
\usepackage{polyglossia}
\setmainlanguage{french}
\setotherlanguage{english}
\usepackage{csquotes}
\usepackage{amsmath,amssymb,amsthm}
\usepackage{mathtools}
\usepackage{hyperxmp}
\usepackage{minted}
\usetheme[progressbar=frametitle]{metropolis}
\usepackage{appendixnumberbeamer}
\usepackage{booktabs}
\usepackage[scale=2]{ccicons}
\usepackage[backend=biber, style=numeric, maxbibnames=99]{biblatex}
\bibliography{slides.bib}

\title{OCaml et la programmation fonctionnelle}
\subtitle{Un fabuleux voyage au cœur de l’informatique}
\date{\today}
\author{Antonin Décimo \and Carine Morel \and Pierre Nigron}
\institute{Université Paris Diderot}

\begin{document}

\maketitle

\begin{frame}{Table des matières}
  \setbeamertemplate{section in toc}[sections numbered]
  \tableofcontents[hideallsubsections]
\end{frame}

\section{OCaml~: un langage multi-paradigmes}

\begin{frame}[fragile]{OCaml~: un langage fonctionnel}
  \begin{itemize}
  \item<1-> langage à expressions (et pas à instructions)
  \item<2-> fonctions d'ordre supérieur
    \begin{minted}{ocaml}
      let rec map f l =
      match l with
      | [] -> []
      | h :: t -> f h :: map f t

      let l = map (fun x -> x + 1) [1; 2; 3]
      (* [2; 3; 4] *)
    \end{minted}
  \item<3-> applications partielles
    \begin{minted}{ocaml}
      let f x y = x * x + y * y
      let g = f 2         (* y -> 4 + y * y *)
      let b = g 3         (* 13 *)
    \end{minted}
  \item<4-> filtrage par motifs
    (\begin{english}pattern-matching\end{english}) et types
    algébriques
    \begin{minted}{ocaml}
      type 'a list = Nil | Cons of 'a * list
      type 'a tree = Leaf of 'a
                   | Node of 'a tree * 'a tree
    \end{minted}
  \end{itemize}
\end{frame}

\begin{frame}[fragile]{OCaml~: un langage multi-paradigmes}
  Un langage impératif~!
  \begin{minted}{ocaml}
    let fresh =
      let c = ref (-1) in
      fun () ->
        c := !c + 1;
        !c

    fresh ()  (* - : int = 0 *)
    fresh ()  (* - : int = 1 *)
  \end{minted}

  Un langage à objets~!
\end{frame}

\begin{frame}[fragile]{OCaml~: du typage}
  \begin{itemize}
  \item typage statique (effectué à la compilation)~;
  \item typage fort (pas de transtypage implicite)~;
    \begin{minted}{ocaml}
      let x = 40 + 2.0
      (* Error: This expression has type float but an
         expression was expected of type int *)
      \end{minted}
  \item inférence de type~;
    \begin{minted}{ocaml}
      let l = [1; 2]
      (* val l : int list = [1; 2] *)
    \end{minted}
  \item du polymorphisme.
    \begin{minted}{ocaml}
      let rec map f l =
      match l with
      | [] -> []
      | h :: t -> f h :: map f t;;
      (* val map : ('a -> 'b) -> 'a list -> 'b list = <fun> *)
    \end{minted}
  \end{itemize}
\end{frame}

\begin{frame}{OCaml~: et plus encore~!}
  \begin{itemize}
  \item compilé ou interprété~;
  \item ramasse-miettes~;
  \item un gestionnaire de paquets~;
  \item possibilité de liaison avec du C.
  \end{itemize}
\end{frame}

\section{Le fonctionnel pur}

\begin{frame}[fragile]{Le fonctionnel pur}
  Un programme fonctionnel pur \emph{n'a pas d'effets de bord}
  \begin{minted}{ocaml}
    let x = ref 0
    let f y = !x * y
    f 2 (* 0 *)
    x := 21
    f 2 (* 42 *)
  \end{minted}

  Quelques effets de bords~:
  \begin{itemize}
  \item entrées et sorties (réseau, fichiers, interactions
    utilisateurs, …)~;
  \item les exceptions~;
  \item affectations sur des données mutables.
  \end{itemize}
  Plus généralement~: interaction observable avec le monde extérieur
\end{frame}

\begin{frame}{Le fonctionnel pur, pourquoi~?}
  Comme une fonction mathématique~! La sortie de ne dépend que de
  l'entrée.

  Intérêts principaux~:
  \begin{itemize}
  \item facilité à raisonner sur le programme~;
  \item preuves~! (Coq)
  \item possibilité d'optimisations.
  \end{itemize}
\end{frame}

\begin{frame}{Effets de bord et pureté}
  Mais, Carine, comment faire des effets de bord dans un langage
  fonctionnel pur~?
\end{frame}

\section{Les Monades}

\begin{frame}{Les Monades}
  Qu'est-ce~? À quoi ça sert~?
  \begin{itemize}
  \item une façon d’ajouter des effets de bord dans des langages purs
    (monade IO, monade de non-déterminisme)~;
  \item un style de programmation avancé, efficace, et modulaire
    (monade d'état, monade liste, monade d'erreur)
  \end{itemize}

  Des objets mathématiques (monoïdes dans la 2-catégorie des
  endo-foncteurs)
\end{frame}

\begin{frame}{Les Monades en théorie}
  Soient deux types $A$ et $B$, une monade est~:
  \begin{description}
  \item[un constructeur de type] $M$ qui construit un type monadique
    $M A$~;
  \item[un convertisseur de type]
    $return$ qui encapsule un objet $x$ dans la monade
    \[ return\quad x : A \to M A \]
  \item[un combinateur] $bind$ qui transporte un objet d'une monade à
    une autre en le transformant
    \[ bind\quad mx\quad f : (M A, A \to M B) \to M B \]
  \end{description}
  … satisfaisant certaines lois.
\end{frame}

\begin{frame}{Les lois monadiques}
  % TODO
\end{frame}

\begin{frame}[fragile]{Monade~: une signature}
  \begin{minted}{ocaml}
    module type Monad = sig
      type 'a t
      val return : 'a -> 'a t
      val bind : 'a t -> ('a -> 'b t) -> 'b t
    end
  \end{minted}
\end{frame}

\begin{frame}[fragile]{La Monade identité}
  \begin{minted}{ocaml}
    module type Monad = sig
      type 'a t
      val return : 'a -> 'a t
      val bind : 'a t -> ('a -> 'b t) -> 'b t
    end
  \end{minted}
\end{frame}

% \begin{frame}{La programmation fonctionnelle, le retour}
%   Est-elle réservée aux langages fonctionnels~?

%   \pause{}

%   \begin{itemize}
%   \item C++ \verb|std::functional|, Java lambdas, …
%   \item Rust, Golang, Swift, …
%   % \item C \inputminted{c}{attributes.c}
%   \end{itemize}

%   ExCamera~\cite{fouladi2017encoding}: un encodeur vidéo VP8 en C++,
%   fonctionnel pur.
% \end{frame}

\end{document}
