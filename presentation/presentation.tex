\documentclass[notes]{beamer}
% \documentclass{beamer}

\usepackage[utf8]{inputenc}
\usepackage[T1]{fontenc}

\usepackage{pgfpages}

\usepackage{listings}
\usepackage{listings-rust}
\usepackage{color}
\usepackage{verbatim}
\usepackage{animate}

\usepackage{siunitx}

\usepackage[frenchb]{babel}
\usepackage[left=\flqq, right=\frqq, leftsub=\flqq, rightsub=\frqq]{dirtytalk}

\usetheme{Warsaw}

% \setbeameroption{show notes on second screen}

\makeatletter
\newenvironment{withoutheadline}{
  \setbeamertemplate{headline}[default]
  \def\beamer@entrycode{\vspace*{-\headheight}}
}{}
\makeatother

\setbeamertemplate{footline}
{%
  \leavevmode
  \begin{beamercolorbox}[wd=.5\paperwidth,ht=2.5ex,dp=1.125ex,leftskip=.3cm plus1fill,rightskip=.3cm]{section in head/foot}%
    Vincent BONNEVALLE % TODO mettez vos nom
  \end{beamercolorbox}%
  \begin{beamercolorbox}[wd=.5\paperwidth,ht=2.5ex,dp=1.125ex,leftskip=.3cm,rightskip=.3cm plus1fil]{subsection in head/foot}%
    Présentation du langage Rust \hfill \Tiny{\textbf{\color{white}{\insertframenumber/\inserttotalframenumber}}}
  \end{beamercolorbox}%
}

\begin{document}

\title{Rustc : meilleur ami, meilleur ennemi}
\author{Julien, Hugo, Vincent Bonnevalle} % Mettez vos nom :)
\institute{Université Paris-Didierot}
\date{}
\titlegraphic{\includegraphics[width=.25\textwidth,height=.25\textheight]{logo.pdf}}

\begin{withoutheadline}
  \begin{frame}
    \titlepage{}
  \end{frame}
\end{withoutheadline}

  \section{Introduction}
  \begin{frame}
    \begin{itemize}
    \item zero-cost abstraction
    \item move semantics
    \item guaranted memory safety
    \item program without data races
    \item trait-based generics
    \item pattern matching
    \item type inference
    \item minimal runtime
    \item efficient C bindings
    \end{itemize}
  \end{frame}

\begin{withoutheadline}
  \begin{frame}
    \frametitle{Plan}
    \setcounter{tocdepth}{1}
    \tableofcontents{}
    \setcounter{tocdepth}{2}
  \end{frame}
\end{withoutheadline}

\section{Section}
\begin{frame}
  \tableofcontents[currentsection, hideothersubsections]
\end{frame}

\subsection{Du code}
\begin{frame}[fragile]
  Contexte d'unification :
\begin{lstlisting}[language=Rust]
let c = "Du code !!!"
\end{lstlisting}
\end{frame}
\subsection{Séparation de la diapo en 2}
\begin{frame}
  \begin{columns}
    \begin{column}{0.5\textwidth}
      Une colonne
    \end{column}
    \begin{column}{0.5\textwidth}
      l'autre colonne
    \end{column}
  \end{columns}
\end{frame}

\section{Syntaxe de Rust}
\begin{frame}
\end{frame}
\section{Le typer de Rust}

\begin{frame}
  \tableofcontents[currentsection, hideothersubsections]
\end{frame}
\subsection{Typage}
\begin{frame}[fragile]
  \begin{itemize}
    \only<1> {
    \item Typage fort
    \item Inférence de type
    \item Transtypage et conversion
    }
    \begin{onlyenv}<2>
    \item Typage fort:
      \begin{lstlisting}[language=Rust]
let x:i64 = 5;
let y:u64 = x;
      \end{lstlisting}
      \begin{lstlisting}[language=bash]
> rustc src/main.rs
# ...
--> src/main.rs:3:17
|
3 |     let y:u64 = x;
|                 ^ expected u64, found i64

error: aborting due to previous error
# ...
      \end{lstlisting}
    \end{onlyenv}
    \begin{onlyenv}<3>
    \item Inférence de type:
      \begin{lstlisting}[language=Rust]
struct Foo {
  bar: i64,
}
fn f(a: Vec<Foo>) -> Foo {

}
fn g(a: Vec<Foo>) {

}
fn main() {
  let v = Vec::new();
  g(v);
}
      \end{lstlisting}
    \end{onlyenv}
    \begin{onlyenv}<4>
    \item Transtypage et conversions:
      \begin{itemize}
      \item Conversions d'entier (sans perte):
      \begin{lstlisting}[language=Rust]
let x:u32 = 1 << 25;
let y:u64 = x as u64
      \end{lstlisting}
      \item \texttt{From} et \texttt{Into}:
      \begin{lstlisting}[language=Rust]
struct Number {value: i32,}
impl From<i32> for Number {
    fn from(item: i32) -> Self {
        Number { value: item }
    }
}
// ...
let x:i32 = 64;
let n = Number::from(x);
let y:i32 = n.into();
      \end{lstlisting}
      \end{itemize}
    \end{onlyenv}
  \end{itemize}
\end{frame}
\subsection{Trait et polymorphisme}
\subsubsection{Trait}
\begin{frame}[fragile]
  trait $\approx$ interfaces de java
      \begin{lstlisting}[language=Rust]
pub trait From<T> {
    fn from(T) -> Self;
}
#[derive(Copy, Clone)]
struct Number {value: i32,}
impl From<i32> for Number {
    fn from(item: i32) -> Self {
        Number { value: item }
    }
}
      \end{lstlisting}
\end{frame}
\subsubsection{Polymorphisme}
\begin{frame}[fragile]
  Paramètre générique de \texttt{Rust} $\approx$ \texttt{templates} de
  \texttt{C++} (mais en mieux).\\
  Nom du type générique en \texttt{CamelCase}.
      \begin{lstlisting}[language=Rust]
fn foo<T>(arg: T) {
  ...
}
struct Bar<T>(T) {
  foo: T,
}
impl<T> Bar<T> {
    fn swap(&mut self) {
      // ...
    }
}
      \end{lstlisting}
\end{frame}
\subsubsection{Polymorphisme contraint}
\begin{frame}[fragile]
  Fusion des paramètres génériques et des traits
      \begin{lstlisting}[language=Rust]
fn foo<T: From<i32>>(arg: T) { ... }
      \end{lstlisting}
      ou
      \begin{lstlisting}[language=Rust]
fn foo<T>(arg: T)
  where T: From<i32> { ... }
// ...
struct Char { c:i32, }
let n = Number::from(64);
foo(n);             // OK
foo(Char { c:64 }); // KO
      \end{lstlisting}
\end{frame}
\subsection{\textit{Borrow checker}}
\begin{frame}
\end{frame}
\section{Rust et la parallélisation}
\begin{frame}
\end{frame}

\end{document}
