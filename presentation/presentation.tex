\documentclass[notes]{beamer}
% \documentclass{beamer}

\usepackage[utf8]{inputenc}
\usepackage[T1]{fontenc}

\usepackage{pgfpages}

\usepackage{listings}
\usepackage{listings-rust}
\usepackage{color}
\usepackage{verbatim}
\usepackage{animate}

\usepackage{siunitx}

\usepackage[frenchb]{babel}
\usepackage[left=\flqq, right=\frqq, leftsub=\flqq, rightsub=\frqq]{dirtytalk}

\usetheme{Warsaw}

% \setbeameroption{show notes on second screen}

\makeatletter
\newenvironment{withoutheadline}{
  \setbeamertemplate{headline}[default]
  \def\beamer@entrycode{\vspace*{-\headheight}}
}{}
\makeatother

\setbeamertemplate{footline}
{%
  \leavevmode
  \begin{beamercolorbox}[wd=.5\paperwidth,ht=2.5ex,dp=1.125ex,leftskip=.3cm plus1fill,rightskip=.3cm]{section in head/foot}%
    Juilien, Hugo, Vincent
  \end{beamercolorbox}%
  \begin{beamercolorbox}[wd=.5\paperwidth,ht=2.5ex,dp=1.125ex,leftskip=.3cm,rightskip=.3cm plus1fil]{subsection in head/foot}%
    Présentation du langage Rust \hfill \Tiny{\textbf{\color{white}{\insertframenumber/\inserttotalframenumber}}}
  \end{beamercolorbox}%
}

\begin{document}

\title{Rustc : meilleur ami, meilleur ennemi}
\author{Julien Rolland, Hugo Pompougnac, Vincent Bonnevalle}
\institute{Université Paris-Didierot}
\date{}
\titlegraphic{\includegraphics[width=.25\textwidth,height=.25\textheight]{logo.pdf}}

\begin{withoutheadline}
  \begin{frame}
    \titlepage{}
  \end{frame}
\end{withoutheadline}

  \section{Introduction}
  \begin{frame}
    \begin{itemize}
    \item zero-cost abstraction
    \item move semantics
    \item guaranted memory safety
    \item program without data races
    \item trait-based generics
    \item pattern matching
    \item type inference
    \item minimal runtime
    \item efficient C bindings
    \end{itemize}
  \end{frame}

\begin{withoutheadline}
  \begin{frame}
    \frametitle{Plan}
    \setcounter{tocdepth}{1}
    \tableofcontents{}
    \setcounter{tocdepth}{2}
  \end{frame}
\end{withoutheadline}

\section{Principaux concepts de programmation}
\subsection{Variable et mutabilité}
\begin{frame}[fragile]
Par défaut, les variables dans Rust sont immutables.
  \begin{lstlisting}[language=rust]
{
    let x = 5;
    x = 6;
}
  \end{lstlisting}
  \begin{lstlisting}[language=bash]
error[E0384]: cannot assign twice
  to immutable variable `x`
 --> src/main.rs:4:5
  |
2 |     let x = 5;
  |         - first assignment to `x`
3 |     x = 6;
  |     ^^^^^ cannot assign twice to immutable
               variable
  \end{lstlisting}
\end{frame}

\begin{frame}[fragile]
  Le mot-clé \texttt{mut} permet de déclarer une variable mutable.
  \begin{lstlisting}[language=rust]
{
    let mut x = 5;
    x = 6;
}
  \end{lstlisting}

  \begin{lstlisting}[language=bash]
> cargo build
Compiling variables v0.1.0
(file:///projects/variables)
Finished dev [unoptimized + debuginfo]
target(s) in 0.30 secs
  \end{lstlisting}
\end{frame}

\begin{frame}[fragile]
  \frametitle{Constantes}
Les constantes sont toujours immutables et ne peuvent pas utilisées avec le mot-clé `mut`.
\begin{lstlisting}[language=rust]
const MAX_POINTS: u32 = 100_000;
\end{lstlisting}
\end{frame}

\begin{frame}[fragile]
  \frametitle{Masquage de nom}
On peut déclarer une nouvelle variable du même nom qu'une autre la précédent, ce qui cache l'ancienne valeur associée à ce nom dans la zone d'éxecution courante.
\begin{lstlisting}[language=rust]
let x = 5;
let x = x + 1;
\end{lstlisting}
Ce concept permet notamment de changer le type de la valeur associée à un nom au fil du programme, ce qui dans certains cas peut être très pratique et plus lisible.
\begin{lstlisting}[language=rust]
let spaces = "   ";
let spaces = spaces.len();
\end{lstlisting}
\end{frame}

\subsection{Types de données primitifs}
\begin{frame}[fragile]
  \frametitle{Types simples}
  \begin{lstlisting}
{
    let i: i8 = 127; // i16, i32, i64, i128, isize
    let u: u8 = 255; // u16, u32, u64, u128, usize
    
    let f: f32 = 5.0; // f64
    
    let b: bool = true;
    
    let c: char = 'c';
    let heart_eyed_cat = '😻';
}
  \end{lstlisting}
\end{frame}

\begin{frame}[fragile]
  \frametitle{Types composés}
On peut définir des tuples de valeurs ayant des types distincts.
  \begin{lstlisting}[language=rust]
{
    let tup: (i32, f64, u8) = (500, 6.4, 1);

    let (x, y, z) = tup;
    
    let five_hundred = tup.0;
    let six_point_four = tup.1;
    let one = tup.2;
}
  \end{lstlisting}
\end{frame}

\begin{frame}[fragile]
  \frametitle{Types composés}
  \begin{lstlisting}[language=rust]
{
    let a: [i32; 5] = [1, 2, 3, 4, 5];

    let element = a[10];
}
  \end{lstlisting}
  \begin{lstlisting}[language=bash, basicstyle=\tiny{}]
> cargo run
   Compiling arrays v0.1.0 (file:///projects/arrays)
    Finished dev [unoptimized + debuginfo] target(s) in 0.31 secs
     Running `target/debug/arrays`
thread '<main>' panicked at 'index out of bounds: the len is 5 but the index is
 10', src/main.rs:6
note: Run with `RUST_BACKTRACE=1` for a backtrace.
  \end{lstlisting}
\end{frame}

\subsection{Fonctions}
\begin{frame}[fragile]
  La fonction \textit{main} est le point d'entrée de tout programme Rust.
Chaque fonction est constituée d'une suite de déclarations optionnellement suivie par une expression dont le résultat sera la valeur de retour.
\begin{lstlisting}[language=rust]
fn main() {
    let x = plus_one(5);

    println!("The value of x is: {}", x);
}

fn plus_one(x: i32) -> i32 {
    x + 1
}
\end{lstlisting}  
\end{frame}

\begin{frame}[fragile]
  Une fonction ne se terminant pas par une expression renvoie le type \texttt{()} correspondant au tuple vide.
  \begin{lstlisting}[language=rust]
error[E0308]: mismatched types
 --> src/main.rs:7:28
  |
7 |   fn plus_one(x: i32) -> i32 {
  |  ____________________________^
8 | |     x + 1;
  | |          - help: consider removing this semicolon
9 | | }
  | |_^ expected i32, found ()
  |
  = note: expected type `i32`
             found type `()`
  \end{lstlisting}
\end{frame}

\subsection{Structures de contrôle}
\begin{frame}[fragile]
  \frametitle{L'expression \texttt{if}}
Structure conditionnelle classique qui peut être utilisée comme une déclaration ou une expression.
\begin{lstlisting}[language=rust]
{
    let x = 5;
    let number = if x < 1 {
        0
    } else if x > 10 {
        1
    } else {
        2
    };
}
\end{lstlisting}  
\end{frame}
\begin{frame}[fragile]
  \frametitle{Les boucles}
La boucle \texttt{ loop } inconditionnelle
  \begin{lstlisting}[language=rust]
{
    let mut counter = 0;

    let result = loop {
        counter += 1;

        if counter == 10 {
            break counter * 2;
        }
    };
}
  \end{lstlisting}
\end{frame}
\begin{frame}[fragile]
  \frametitle{Les boucles}
  \begin{lstlisting}[language=rust]
let a = [10, 20, 30, 40, 50];

let mut index = 0;
while index < 5 {
    println!("the value is: {}", a[index]);
    index = index + 1;
}

for element in a.iter() {
    println!("the value is: {}", element);
}
for index in 0..5 {
    println!("the value is: {}", a[index]);
}
  \end{lstlisting}
\end{frame}
\subsection{Les structures}

\begin{frame}[fragile]
  \frametitle{Définition et instanciation}
Une structure est une composition de données où chaque champs est nommé (l'ordre des membres n'est donc pas significatif).
  \begin{lstlisting}[language=rust]
struct Rectangle {
    width: u32,
    height: u32,
}
let r = Rectangle {
    height: 25,
    width: 10,
}
  \end{lstlisting}
\end{frame}

\begin{frame}[fragile]
  \frametitle{Définition de méthodes}
\begin{lstlisting}[language=rust, basicstyle=\tiny{}]
impl Rectangle {
    fn area(&self) -> u32 {
        self.width * self.height
    }
}

fn main() {
    let rect1 = Rectangle { width: 30, height: 50 };

    println!(
        "The area of the rectangle is {} square pixels.",
        rect1.area()
    );
}
\end{lstlisting}
Chaque méthode a pour premier paramêtre le mot-clé \texttt{self}, ce qui permet d'utiliser les champs de la structure depuis laquelle elle est invoquée.
\end{frame}

\begin{frame}[fragile]
  \frametitle{Définition de fonctions associées}
On peut également définir des fonctions qui ne s'exécutent pas à partir d'une instance de structure.
\begin{lstlisting}[language=rust]
impl Rectangle {
    fn square(size: u32) -> Rectangle {
        Rectangle { width: size, height: size }
    }
}

let sq = Rectangle::square(3);
\end{lstlisting}  
\end{frame}

\subsection{Les types énumérés}

\begin{frame}[fragile]
  \frametitle{Définition et instanciation}
  \begin{lstlisting}[language=rust]
enum Message {
    Quit,
    Move { x: i32, y: i32 },
    Write(String),
    ChangeColor(u8, u8, u8),
}

let message = Message::ChangeColor(0,255,255);
  \end{lstlisting}
\end{frame}

\begin{frame}[fragile]
  \frametitle{Le type \texttt{enum} Option}
  \begin{lstlisting}[language=rust]
enum Option<T> {
    Some(T),
    None,
}
  \end{lstlisting}
\end{frame}

\begin{frame}[fragile]
  \frametitle{Le \textit{pattern matching}}
  \begin{lstlisting}[language=rust, basicstyle=\small{}]
enum Coin {
    Penny,
    Nickel,
    Dime,
    Quarter,
}

fn value_in_cents(coin: Coin) -> u32 {
    match coin {
        Coin::Penny => 1,
        Coin::Nickel => 5,
        Coin::Dime => 10,
        Coin::Quarter => 25,
    }
}
  \end{lstlisting}
\end{frame}

\begin{frame}[fragile]
  \frametitle{Éxhaustivité}
  \begin{lstlisting}[language=rust]
{
    let value = 8;
    match value {
       1 => println!("one"),
       3 => println!("three"),
       5 => println!("five"),
       7 => println!("seven"),
       _ => (),
    }
}
  \end{lstlisting}
\end{frame}


\begin{frame}[fragile]
  \frametitle{Sucre syntaxique}
  \begin{lstlisting}[language=rust]
let value = Some(14);

match some_u8_value {
    Some(3) => println!("three"),
    _ => (),
}

if let Some(3) = value {
    println!("three");
}
  \end{lstlisting}
\end{frame}

\subsection{La gestion des erreurs}

\begin{frame}[fragile]
  \frametitle{Ne pas paniquer !}
  \begin{lstlisting}[language=rust]
fn main() {
    panic!("crash and burn");
}
  \end{lstlisting}
  \begin{lstlisting}[language=bash, basicstyle=\small{}]
> cargo run
   Compiling panic v0.1.0 (file:///projects/panic)
    Finished dev [unoptimized + debuginfo]
        target(s) in 0.25 secs
     Running `target/debug/panic`
thread 'main' panicked at 'crash and burn',
        src/main.rs:2:4
note: Run with `RUST_BACKTRACE=1` for a backtrace.
  \end{lstlisting}
\end{frame}

\begin{frame}[fragile]
  \frametitle{Erreurs non rédhibitoires}
  \begin{lstlisting}[language=rust, basicstyle=\small{}]
enum Result<T, E> {
    Ok(T),
    Err(E),
}

fn main() {
    let f = File::open("hello.txt");
    let f = match f {
        Ok(file) => file,
        Err(error) => {
            panic!("{:?}", error)
        },
    };
}
  \end{lstlisting}
\end{frame}

\begin{frame}[fragile]
  \frametitle{Raccourcis}
  \begin{lstlisting}[language=rust]
fn main() {
    let f = File::open("hello.txt").unwrap();
}

fn main() {
    let f = File::open("hello.txt")
        .expect("Failed to open hello.txt");
}
  \end{lstlisting}
\end{frame}

\begin{frame}[fragile]
  \frametitle{Propagation des erreurs}
  \begin{lstlisting}[language=rust, basicstyle=\small{}]
fn read_username_from_file() -> Result<String, io::Error> {
    let f = File::open("hello.txt");

    let mut f = match f {
        Ok(file) => file,
        Err(e) => return Err(e),
    };

    let mut s = String::new();

    match f.read_to_string(&mut s) {
        Ok(_) => Ok(s),
        Err(e) => Err(e),
    }
}
  \end{lstlisting}
\end{frame}

\begin{frame}[fragile]
  \begin{lstlisting}[language=rust]
fn read_username_from_file()
        -> Result<String, io::Error> {
    let mut f = File::open("hello.txt")?;
    let mut s = String::new();
    f.read_to_string(&mut s)?;
    Ok(s)
}
  \end{lstlisting}
\end{frame}

\section{Le typer de Rust}

\begin{frame}
  \tableofcontents[currentsection, hideothersubsections]
\end{frame}
\subsection{Typage}
\begin{frame}[fragile]
  \begin{itemize}
    \only<1> {
    \item Typage fort
    \item Inférence de type
    \item Transtypage et conversion
    }
    \begin{onlyenv}<2>
    \item Typage fort:
      \begin{lstlisting}[language=Rust]
let x:i64 = 5;
let y:u64 = x;
      \end{lstlisting}
      \begin{lstlisting}[language=bash]
> rustc src/main.rs
# ...
--> src/main.rs:3:17
|
3 |     let y:u64 = x;
|                 ^ expected u64, found i64

error: aborting due to previous error
# ...
      \end{lstlisting}
    \end{onlyenv}
    \begin{onlyenv}<3>
    \item Inférence de type:
      \begin{lstlisting}[language=Rust]
struct Foo {
  bar: i64,
}
fn f(a: Vec<Foo>) -> Foo {

}
fn g(a: Vec<Foo>) {

}
fn main() {
  let v = Vec::new();
  g(v);
}
      \end{lstlisting}
    \end{onlyenv}
    \begin{onlyenv}<4>
    \item Transtypage et conversions:
      \begin{itemize}
      \item Conversions d'entier (sans perte):
      \begin{lstlisting}[language=Rust]
let x:u32 = 1 << 25;
let y:u64 = x as u64
      \end{lstlisting}
      \item \texttt{From} et \texttt{Into}:
      \begin{lstlisting}[language=Rust]
struct Number {value: i32,}
impl From<i32> for Number {
    fn from(item: i32) -> Self {
        Number { value: item }
    }
}
// ...
let x:i32 = 64;
let n = Number::from(x);
let y:i32 = n.into();
      \end{lstlisting}
      \end{itemize}
    \end{onlyenv}
  \end{itemize}
\end{frame}
\subsection{Trait et polymorphisme}
\subsubsection{Trait}
\begin{frame}[fragile]
  trait $\approx$ interfaces de java
      \begin{lstlisting}[language=Rust]
pub trait From<T> {
    fn from(T) -> Self;
}
#[derive(Copy, Clone)]
struct Number {value: i32,}
impl From<i32> for Number {
    fn from(item: i32) -> Self {
        Number { value: item }
    }
}
      \end{lstlisting}
\end{frame}
\subsubsection{Polymorphisme}
\begin{frame}[fragile]
  Paramètre générique de \texttt{Rust} $\approx$ \texttt{templates} de
  \texttt{C++} (mais en mieux).\\
  Nom du type générique en \texttt{CamelCase}.
      \begin{lstlisting}[language=Rust]
fn foo<T>(arg: T) {
  ...
}
struct Bar<T>(T) {
  foo: T,
}
impl<T> Bar<T> {
    fn swap(&mut self) {
      // ...
    }
}
      \end{lstlisting}
\end{frame}
\subsubsection{Polymorphisme contraint}
\begin{frame}[fragile]
  Fusion des paramètres génériques et des traits
      \begin{lstlisting}[language=Rust]
fn foo<T: From<i32>>(arg: T) { ... }
      \end{lstlisting}
      ou
      \begin{lstlisting}[language=Rust]
fn foo<T>(arg: T)
  where T: From<i32> { ... }
// ...
struct Char { c:i32, }
let n = Number::from(64);
foo(n);             // OK
foo(Char { c:64 }); // KO
      \end{lstlisting}
\end{frame}
\subsection{\textit{Borrow checker}}
\begin{frame}
\end{frame}
\section{Rust : le parallélisme "sans peur et sans reproche"}
\subsection{Des promesses alléchantes}
\begin{frame}{Quel est le problème avec le parallélisme ?}
  Dans la plupart des langages impératifs (et singulièrement en C) :
  \begin{itemize}
  \item Il est très simple d'engendrer un bug avec l'écriture concurrente d'un emplacement mémoire insuffisamment protégé.
  \item Il est très compliqué de retrouver le morceau de code qui cause le bug, surtout si le programme ne crashe pas mais corrompt simplement les données.
  \item C'est d'autant plus vrai que la survenue ou non du bug dépend de chaque exécution.
  \end{itemize}
\end{frame}

\begin{frame}
  \frametitle{Rust à la rescousse !}
  Rust propose donc de vérifier à la compilation qu'un de ces bugs ne peut pas être introduit :
  \begin{itemize}
  \item Pour que le code compile, il faut qu'il soit thread-safe (et le compilateur indique à quelle ligne il ne l'est pas).
  \item Une série de mécanismes, dont la plupart reposent sur le borrow-checker de Rust, sont fournis pour apporter ces garanties statiques.
  \item Ils peuvent être désactivés en insérant le code problématique dans un bloc unsafe, mais dans ce cas-là c'est le programmeur lui-même qui choisit de bugger son code.
  \end{itemize}
\end{frame}

\subsection{Pour commencer : une variable partagée sans protection}
\begin{frame}[fragile]
  \frametitle{Dans le monde du C}
  Considérons le code C suivant, consistant en la pire manière possible de faire du parallélisme :
  \begin{lstlisting}[language=C, basicstyle=\tiny{}]
#include <pthread.h>
#include <stdio.h>

#define NUM_THREADS 100

int share_var = 0 ;

void * thread_affect(void* arg) {
  share_var = 0 ;
}

void * thread_div(void* arg) {
  if (share_var != 0) {
    fprintf(stderr, "%s", "Suis-je en train de diviser par zero ?\n"); 
    share_var = 800/share_var + 12 ;
  }
  else {
    share_var = 8 ;
  }
}
  \end{lstlisting}
\end{frame}
\begin{frame}[fragile]
  \begin{lstlisting}[language=C, basicstyle=\tiny{}]
int main() {
  pthread_t threads[NUM_THREADS] ;

  for (int i = 0 ; i < NUM_THREADS ; i++) {
    if (i % 2 == 0) {
    pthread_create(&threads[i], NULL, thread_affect, NULL) ;
  }
  else {
    pthread_create(&threads[i], NULL, thread_div, NULL) ;
  }
}

for (int i = 0 ; i < NUM_THREADS ; i++) {
  pthread_join(threads[i], NULL) ;
}

printf("%d\n",share_var) ;
}
\end{lstlisting}  
\end{frame}
\begin{frame}[fragile]
  La compilation et l'exécution répétée de ce code produit les sorties suivantes :
  \begin{lstlisting}[language=bash, basicstyle=\tiny{}]
> gcc unsafe_concurrency.c -lpthread
> ./a.out
> ./a.out
Suis-je en train de diviser par zero ?
8
> ./a.out
Suis-je en train de diviser par zero ?
Suis-je en train de diviser par zero ?
Exception en point flottant
> ./a.out
Suis-je en train de diviser par zero ?
Exception en point flottant
  \end{lstlisting}
  La raison est simple : une fois qu'on a vérifié que \texttt{share\_var} est différent de 0, on n'a aucune garantie que c'est toujours le cas deux lignes plus loin.
\end{frame}

\begin{frame}[fragile]
  \frametitle{Dans le monde de \texttt{Rust}}
  \begin{lstlisting}[language=rust, basicstyle=\tiny{}]
const NTHREADS: i32 = 100;
static mut share_var: i32 = 0 ;
fn main() {
    let mut threads = vec![];
    for i in 0..NTHREADS {
        if i % 2 == 0 {
            threads.push(thread::spawn(|| share_var = 0));
        }
        else {
            threads.push(thread::spawn(|| {
                if share_var != 0 {
                    eprintln!("Suis-je en train de diviser par zero ?\n");
                    share_var = 800/share_var + 12 ;
                }
                else {
                 share_var = 8 ;
                }
            }));
        }
    }
    for child in threads { child.join(); }
    println!("{}",share_var);
}
  \end{lstlisting} 
\end{frame}
\begin{frame}[fragile]
Il ne passe même pas la compilation :

\begin{lstlisting}[language=bash]
> rustc unsafe_concurrency.rs
error[E0133]: use of mutable static is unsafe
  and requires unsafe function or block
--> unsafe_concurrency.rs:11:48
error[E0133]: use of mutable static is unsafe
  and requires unsafe function or block
...
\end{lstlisting}
\end{frame}
\begin{frame}[fragile]
  \frametitle{Nouvel essai}
  Il semble y avoir un problème avec les variables globales (\texttt{static}). Mais que se passe-t-il si nous essayons de "piéger" le compilateur en déclarant \texttt{share\_var} localement et en cherchant à la modifier via la clôture d'un thread (avec un code moins volumineux) ?
  \begin{lstlisting}[language=rust]
use std::thread;

const NTHREADS: i32 = 100;

fn main() {
    let mut share_var: i32 = 0 ;
    thread::spawn(|| share_var = 0);
}
  \end{lstlisting}
\end{frame}
\begin{frame}[fragile]
Voyons ce que nous dit le compilateur :
\begin{lstlisting}[language=bash]
> rustc unsafe_concurrency2.rs
error[E0373]: closure may outlive
the current function,
but it borrows share_var,
which is owned by the current function
\end{lstlisting}
Pour passer la compilation, on peut forcer la propriété de \texttt{share\_var} par le mot-clé \texttt{move} :
\begin{lstlisting}[language=rust]
thread::spawn(move || share_var = 0);
\end{lstlisting}

Mais ce mot clé ne fait qu'un passage par valeur : on ne modifie plus qu'une
variable locale au
\textit{thread}, et non la variable partagée entre les \textit{threads} \dots
\end{frame}

\begin{frame}
  \frametitle{Le premier commandement de \texttt{Rust} : "thread isolation"}
Jusqu'ici, on constate donc que Rust interdit l'écriture concurrente de données de deux manières :
\begin{itemize}
\item En interdisant l'écriture de variables globales.
\item En s'assurant, via le \textit{borrow checker}, qu'une variable accessible en écriture ne soit jamais possédée par plus d'une fonction (et donc d'un \textit{thread}).
\end{itemize}

Les \textit{threads} étant statiquement isolés les uns des autres, il est impossible que l'un d'entre eux corrompe les données utilisées par un autre. Mais à ce compte-là, comment peuvent-ils partager des données ? En les protégeant !
\end{frame}

\subsection{Les mutex : chacun son tour !}
\begin{frame}[fragile]
  \frametitle{Passer les fourches caudines du compilateur}
Spontanément, pour protéger des données partagées, on les verrouille avec un Mutex garantissant l'absence d'accès concurrents :
\begin{lstlisting}[language=rust, basicstyle=\tiny{}]
use std::sync::{Arc, Mutex};
use std::thread;
const nthreads: i32 = 100;
fn main() {
    let share_var = Arc::new(Mutex::new(0)) ;
    let mut threads = vec![];
    // Traitement des threads
    for child in threads { child.join(); }
    println!("{}",*share_var.lock().unwrap());
}
\end{lstlisting}
\end{frame}
\begin{frame}[fragile]
  \begin{lstlisting}[language=rust, basicstyle=\tiny{}]
    for i in 0..nthreads {
        let share_var = share_var.clone() ;
        if i % 2 == 0 {
            threads.push(thread::spawn(move || {
                let mut data = share_var.lock().unwrap() ;
                *data = 0 ;
            })) ;
        }
        else {
            threads.push(thread::spawn(move || {
                let mut data = share_var.lock().unwrap() ;
                if *data != 0 {
                    eprint!("suis-je en train de diviser par zero ?\n");
                    *data = 800/(*data) + 12 ;
                }
                else {
                 *data = 8 ;
                }
            }));
        }
    }
  \end{lstlisting}
\end{frame}
\begin{frame}[fragile]
  \frametitle{\textit{"Lock data, no code"}}
Que se passe-t-il ici ?
\begin{itemize}
\item Pour prendre la propriété sur la variable partagée, il faut demander un \textit{lock}.
\item Le \textit{lock} est relâché quand on sort de la portée où il est demandé.
\item On ne peut donc pas accéder aux données en même temps qu'un autre \textit{thread}.
\item Le \textit{mutex} est transmis par valeur aux \textit{threads}, mais peu importe, puisque le verrou sur lequel il pointe est unique (c'est un pointeur de type \textit{Arc}, nous y reviendrons).
\end{itemize}
\end{frame}
\begin{frame}[fragile]
Il est donc absolument impossible d'accéder à la variable dans une section du
code qui n'est pas protégée. En \texttt{C}, au contraire, on peut écrire : 
\begin{lstlisting}[language=C]
void * thread(void* arg) {
    pthread_mutex_lock(mut);
    share_var = 0 ;
    pthread_mutex_unlock(mut);
    /* ... */
    share_var = 5 ;
    /* ... */
}
\end{lstlisting}
Le \texttt{C} protège ("dynamiquement") la portion de code encadrée d'un \texttt{mutex} : \texttt{Rust} protège (statiquement) la variable partagée quelle que soit sa position dans le code.
\end{frame}

\subsection{Send et !Send}
\begin{frame}[fragile]
  \frametitle{Robustesse et typage en \texttt{Rust}}
  Jusqu'ici, nous transférons des données d'un thread à l'autre sans les corrompre. Comme dans tous les langages, ce n'est pas possible en toute généralité. Certains types de pointeurs, par exemple, ne garantissent pas l'atomicité de leur déréférencement. C'est le cas des pointeurs \texttt{Rc} (contrairement aux pointeurs \texttt{Arc} que nous utilisons dans le cas des \texttt{Mutex}) :
  \begin{lstlisting}[language=rust, basicstyle=\small{}]
const NTHREADS: i32 = 100;
fn main() {
    let share_var = Rc::new(Mutex::new(0)) ;
    thread::spawn(move || {
        let mut data = share_var.lock().unwrap() ;
        *data = 0 ;
    }) ;
}
  \end{lstlisting}
\end{frame}
\begin{frame}[fragile]
Donne l'erreur de compilation suivante :
\begin{lstlisting}[language=bash]
> rustc unatomic.rs
error[E0277]: `std::rc::Rc<std::sync::Mutex<i32>>`
   cannot be sent between threads safely 
--> unatomic.rs:9:5
\end{lstlisting}
\end{frame}

\begin{frame}
  \frametitle{\textit{Thread safety isn't just documentation ; it's law}}
Ainsi : 
\begin{itemize}
\item  \texttt{Rust} permet d'utiliser librement les différents types de données sans avoir besoin de se documenter sur chaque type pour identifier un accès concurrent (en l'occurence à un pointeur).
\item  On dit des types \textit{thread-safe} qu'il sont \texttt{Send}, et des autres qu'ils sont \texttt{!Send}.
\item  À l'inverse, en \texttt{C++} par exemple, l'usage d'un \textit{smart-pointer} (spécifié pour être \textit{thread-safe}) ou d'un pointeur "nu" (dont la spécification ne garantit rien) est indifférent du point de vue des erreurs de compilation.
\end{itemize}
\end{frame}
\subsection{Pour aller plus loin : la communication par message}
\begin{frame}
  \frametitle{Pourquoi la communication par message ?}
  Il est courant de synchroniser les données de plusieurs \textit{threads} en passant par des \textit{mutex} : mais n'est-il pas plus simple et plus intuitif de leur permettre de s'envoyer des messages ?

En effet, on se représente l'interdiction des écritures concurrentes de manière moins artificielle :
\begin{itemize}
\item  S'il faut avoir reçu une donnée pour la manipuler.
\item  S'il faut envoyer une donnée pour qu'un autre puisse la manipuler.
\item  Si on ne peut plus modifier une donnée qu'on a envoyée.
\end{itemize}

Avec un modèle de ce type, notre variable fonctionne comme une feuille de papier : seul celui qui l'a dans les mains peut agir dessus... Et à nouveau, on devine que le borrow-checker va permettre de rendre ce mécanisme très robuste. C'est de cette manière que fonctionnent les \textit{channels}.
\end{frame}

\begin{frame}[fragile]
  \frametitle{Utiliser les channels : thread isolation++}
  Un \textit{channel} consiste en une paire (un récepteur et un émetteur) que l'on construit en faisant appel à la fonction :
  Ce \texttt{Sender} et ce \texttt{Receiver}, partagés, permettent à plusieurs \textit{threads} de communiquer entre eux.
  \begin{lstlisting}[language=rust]
pub fn send(&self, t:T) -> Result<(), SendError<T>>
  \end{lstlisting}
  Et on reçoit une variable en utilisant la méthode suivante du \texttt{Receiver}, qui lui en reprend la propriété :
  \begin{lstlisting}[language=rust]
pub fn recv(&self) -> Result<T, RecvError>
  \end{lstlisting}
La réception est évidemment bloquante.

Ainsi, la question de la protection de la mémoire partagée ne se pose plus (ou très différemment) : en effet, une telle variable n'existe tout simplement pas du point de vue d'un \textit{thread} tant qu'elle ne lui a pas été transférée via un \textit{channel}.
\end{frame}

\begin{frame}[fragile]
  \frametitle{Un exemple : la factorielle}
  \begin{lstlisting}[language=rust, basicstyle=\tiny{}]
use std::thread;
use std::sync::mpsc::channel;

const N : i32 = 10 ;

fn main() {
    let (tx, rx) = channel();
    for i in 0..N {
        let tx = tx.clone();
        thread::spawn(move|| {
            let mut n = i ;
            n = n + 1 ;
            tx.send(n).unwrap();
            n = n * 100 ;
        });
    }

    let mut res = 1 ;
    for _ in 0..N {
        res = res * rx.recv().unwrap();
    }
    println!("{}",res);
}
  \end{lstlisting}
  
\end{frame}

\begin{frame}
Quelques remarques :
\begin{itemize}
\item  Cet exemple est artificiel : le calcul pour obtenir n est trivial. Imaginons qu'on l'obtient au prix d'opérations coûteuses, comme la lecture dans un fichier.
\item  La modification de n après l'envoi n'a aucun impact sur la valeur envoyée (partagée).
\item  Le thread principal ne finit pas tant qu'il n'a pas reçu ses 10 entiers, et il ne les reçoit pas tant que les threads fils ne les lui ont pas envoyé.
\item  Les channels standard de Rust permettent plusieurs écrivains, mais un seul lecteur... Notre exemple du début n'est donc pas si facile à écrire de cette manière.
\end{itemize}
\end{frame}

\begin{frame}
  \frametitle{Conclusion}
  \begin{itemize}
  \item  Ce tour d'horizon est évidemment minimal, et d'autres mécanismes existent : par exemple, la possibilité pour un \textit{thread} de partager sa pile avec ses enfants sans courir de risque.
  \item  Néanmoins, il permet de toucher du doigt l'un des c\oe{}urs de la conception de \texttt{Rust} : le compilateur, en tirant profit du \textit{borrow-checking}, interdit statiquement tout accès concurrent impliquant une écriture.
  \item  Les principes présentés ici sont issus de https://blog.rust-lang.org/2015/04/10/Fearless-Concurrency.html ; le détail des différents types de données que nous manipulons sont tirés de la documentation officielle ; et les exemples sont écrits par nos soins.
  \end{itemize}
\end{frame}





\end{document}
